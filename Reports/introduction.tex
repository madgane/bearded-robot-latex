The current wireless standards are moving towards the packet switched networks to improve the system performance by reducing the delay involved in the circuit switched networks. The \ac{RAT} aim at achieving higher data rates due to the demand from the higher layers and also the competition involved with other \ac{RAT}s. The potential candidate to increase the throughput is by multiplexing users spatially which is called as \ac{MU} \ac{MIMO}. In order for MU-MIMO to provide the multiplexing gain, users with linearly independent channel vectors must be scheduled. The linearly independent vectors are selected in order to employ the linear receivers at the user terminals and the linear precoders at the transmitters. The selection of users with such a constraint is mainly done by the schedulers in order to utilize the wireless resources efficiently. Precoder design can either be based on \ac{ZF} as discussed in \cite{spencer2004zero}, \cite{wiesel2008zero} or based on optimization schemes with the sum rate maximizing or power minimizing objective.

User selection for MU-MIMO can be based on a successive projection which projects the users over the null space of the chosen user channel vectors \cite{shen2006low}. A Search algorithm based on \ac{SUS} using \ac{GS} orthogonalization was proposed in \cite{sus2006zfbf}. Algorithms based on \ac{BD} of the stacked channel vectors by projecting successively onto the null space were discussed in \cite{shen2006low} and \cite{youtuan2007improved}.

Precoder design in an iterative manner was proposed in \cite{traniterative} for BD based user search. The paper also addressed the complexity issues involved in diagonalizing the large concatenated matrix. The user selection schemes for MU-MIMO based transmission scheme were classified and analyzed briefly in \cite{zhang2007user}, which aimed at minimizing the average beamforming power. Scheduling based on genetic search, which included different criteria to be optimized, was discussed in \cite{genetic_search}. Scheduling for multiple antenna receivers based on iteratively updating the receiver was discussed in \cite{antti_user_selection}.

The user selection based on maximizing the volume subtended by the user channel vectors were discussed in \cite{ko2012determinant} and \cite{jin2010novel}. The volume based selection performs identical to the \ac{SUS} scheme, as both are based on \ac{GS} orthogonalization procedure. The sum rate maximizing user selection provides stable queues, when the path loss between the base station and users are normalized \cite{queuevsinfo}.

In this paper, we propose scheduling schemes which can provide improved sum rate over the existing ones with the increased complexity. We also discuss scheduling algorithms which approximate the null space by independent vector projections that perform well with the existing schemes with significant reduction in the complexity. The complexities involved in the proposed and the existing schemes are provided with the figure comparing the number of operations involved by varying the number of users in the system.

