
The scheduling of the users for \ac{MU} \ac{MIMO} transmission requires the knowledge of the channel conditions of each user in order to achieve considerable gain over \ac{SU} \ac{MIMO}. The \ac{MU}-\ac{MIMO} transmission requires user channel vectors to be linearly independent to provide interference-free transmission for the multiplexed users with the help of precoders.

There are numerous papers discussing the scheduling of users for \ac{MU}-\ac{MIMO} transmission with the \ac{ZF} \ac{BF} design. The precoder design based on \ac{ZF}-\ac{BF} provides efficient de-coupling of the transmitted streams at the receiver by zeroing the interference caused by other streams over the desired channel \cite{spencer2004zero,wiesel2008zero}. The precoders are designed for the set of users whose data streams are to be multiplexed at the given scheduling instant. The \ac{MU}-\ac{MIMO} user set is identified from the available pool of users based on \ac{QoS} constraints and the linear independency of the channel vectors.
The selection of users based on linearly independent channel vectors were well studied in the literature. The selection based on the \ac{GS} algorithm as \ac{SUS} algorithm was discussed in \cite{sus2006zfbf} and the extension with the multi-antenna receivers as considered in \cite{antti_user_selection}.  Algorithms using \ac{BD} of the stacked channel vectors by successively projecting onto the null space as discussed in \cite{shen2006low} and \cite{youtuan2007improved}. Selection of users based on stochastic algorithms were considered in \cite{genetic_search} using \ac{PF} or sum rate maximization objectives. The user selection using the channel volume enclosure were considered in the scheduling for \ac{MU-MIMO} in \cite{ko2012determinant} and \cite{jin2010novel}. The selection based on volume enclosure provides similar selection based on \ac{SUS} algorithm.

The joint precoder design and scheduling were carried out for \ac{MU}-\ac{MIMO} transmission using the interference leakage formulation \cite{sadek} and \cite{leakage}. The selection was carried out in a greedy manner using the \ac{SLNR} metric and the precoders were designed based on the maximum eigen-vector of the matrices involved \ac{SLNR} calculation. The iterative precoder design considered in \cite{traniterative} was based on the \ac{BD} of the channel vectors of stacked user channels and the precoders were designed after each selection. The precoder design using \ac{W-MMSE} equivalent for \ac{WSRM} was considered using the alternating optimization method for obtaining the precoder in an iterative manner. The overloaded design in which the number of users are greater than the available spatial \ac{DoF}, the \ac{W-MMSE} scheme performs joint scheduling and the precoder design by providing zero powered precoders for nonscheduled users \cite{wmmse_shi}.

This paper discusses the joint scheduling of cell-edge users across multi-cell scenario in an iterative manner to provide interference-free transmission. Since the users are at the cell-edge are interference limited, the available spatial \ac{DoF} are shared across the transmission user sets of all \ac{BS}s. The proposed joint scheduling schemes are compared with other joint scheduling and precoder design schemes in the literature. The proposed schemes provides a way to use the simple \ac{ZF}-\ac{BF} based precoders to achieve similar performance as of \ac{W-MMSE} scheme.
